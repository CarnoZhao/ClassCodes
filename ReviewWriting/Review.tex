\documentclass[lang=cn]{elegantpaper}

\title{胚胎器官发生图谱构建中单细胞测序的应用}
\author{赵洵}
\institute{中国科学院大学}
\version{0.01}
\date{\today}

\begin{document}
\maketitle

\begin{abstract}
摘要
\keywords{单细胞测序\ 器官发育\ 胚胎发育}
\end{abstract}

\section{引言}

\section{正文}
\subsection{单细胞RNA测序}
DNA测序技术在40年来飞速发展,从最初的Sanger测序到现在第二代和第三代例如Ilummina、PacBio、Nanopore测序技术,已经极大的提高了测序通量并且降低了测序价格\citep{shendure_dna_2017}。RNA测序技术 (RNA-seq) 在2008年首次出现\citep{cloonan_stem_2008},在一年之后,即2009年,出现了单细胞RNA测序 (scRNA-seq)\citep{tang_mrna-seq_2009}。

单细胞RNA测序大致分为四步。首先是将样品分为单个细胞或单个细胞核,再经过逆转录并对cDNA进行PCR扩增,最后进行测序。常见的scRNA-seq测序方法有种,一共可以归为两类,一类为对RNA全长测序 (full-length),一类为基于人工添加标签的测序 (tag-based)\citep{hedlund_single-cell_2018}。在面对不同生物学问题上两种类别各有优劣,全长测序有更高的敏感度和可重复性,标签法测序适用于更廉价的大规模测序\citep{ziegenhain_comparative_2017}。

scRNA-seq基于原有RNA-seq技术

相比于原有的RNA测序技术,scRNA-seq最重要的一点改进即在单个细胞水平的分辨率,原有的技术只能对细胞群体进行基因表达。

\section{讨论}
\subsection{讨论第一部分}
    讨论第一部分内容
\section{结论}
这是结论
\bibliography{ReviewWriting}
\end{document}
