\documentclass[lang=cn]{elegantpaper}

\title{胚胎器官发生图谱构建中单细胞测序的应用}
\author{赵洵}
\institute{中国科学院大学}
\version{0.01}
\date{\today}

\begin{document}
\maketitle

\begin{abstract}
摘要
\keywords{单细胞测序\ 器官发育\ 胚胎发育}
\end{abstract}

\section{引言}
器官发生是胚胎发育过程中自原肠胚形成直到出生的一段时期\citep{gilbert_developmental_2017},原肠胚外胚层、中胚层、内胚层不同位置的细胞逐步发育为不同器官,这个过程需要各类细胞和组织的紧密协调,受到多种生长因子通路的调控,且随着时间不断变化\citep{zorn_vertebrate_2009}。对器官发生过程中单细胞的基因表达分析有着重要意义,一方面,准确区分不同的细胞类型有助于建立完整的细胞间相互调控网络体系\citep{dong_single-cell_2018}。另一方面,为了解器官在正常和病理条件下发育机制的区别,解析胚胎中准确细胞表达模式是至关重要的\citep{belle_tridimensional_2017}。由于早期胚胎多能干细胞具有分化潜力,对器官发生过程中各类干细胞的分化过程的分析会有助于细胞治疗的发展\citep{lancaster_organogenesis_2014},对于特定细胞的表达模式分析,可以诱导其他细胞表达某些基因使其具有其他类型细胞的功能,例如仅诱导成纤维细胞表达四个转录因子即可使其转变为干细胞\citep{takahashi_induction_2006}。

相比于传统研究方法,单细胞RNA测序同时具有高分辨率高通量的优点,还可以综合各类统计分析方法和实验结果,全面分析胚胎器官发育过程中的细胞变化。

\section{单细胞RNA测序}
DNA测序技术在40年来快速发展,从最初的Sanger测序到现在第二代和第三代例如Ilummina、PacBio、Nanopore测序技术,已经极大的提高了测序通量并且降低了测序价格\citep{shendure_dna_2017}。随着表达序列标签(EST)、DNA元件百科(ENCODE)等计划的启动以及对真核细胞转录组的了解日渐加深\citep{birney_identification_2007},人们意识到已有的DNA阵列、标签筛选等方法已经不足以应付真核细胞转录组的复杂性,于是借助高通量测序技术分析转录组数据的RNA测序技术 (RNA-seq) 在2008年首次出现\citep{cloonan_stem_2008},在一年之后,即2009年,一方面为了解决某些实验条件下样品量缺乏的问题,例如早期胚胎发育的研究中并不能获得足量的实验样品,另一方面借助cDNA扩增技术准确性的突破,出现了单细胞RNA测序 (scRNA-seq)\citep{tang_mrna-seq_2009}。

\subsection{测序方法}
单细胞RNA测序大致分为四步。首先是将样品分为单个细胞或单个细胞核,再经过逆转录并对cDNA进行PCR扩增,最后使用第二代高通量测序技术对扩增后的cDNA进行测序。在单细胞分离阶段,可以使用微液滴、微孔芯片、流式细胞术、微移液管技术等将细胞悬液分为单个细胞\citep{hedlund_single-cell_2018}。在逆转录阶段,因为测序只需要mRNA数据,所以需要使用寡聚胸腺嘧啶或随机六位核酸来去除rRNA的影响(占哺乳动物细胞总RNA数的95\%),但是实验发现即使不经过rRNA清除步骤也不会带来显著的rRNA污染,因为常见实验条件还不足以打开rRNA的二级结构\citep{fan_single-cell_2015}。常见的单细胞RNA测序测序方法有多种,一共可以归为两类,一类为对RNA全长测序 (full-length),一类为基于人工添加标签的测序 (tag-based),在面对不同生物学问题上两种类别各有优劣,全长测序有更高的敏感度和可重复性,标签法测序适用于更廉价的大规模测序\citep{ziegenhain_comparative_2017}。

在胚胎器官发生的研究中,Drop-seq\citep{macosko_highly_2015}是一种非常常用的测序方法,Drop-seq属于基于人工添加标签的测序方法。在分离细胞的过程中,每单个细胞会进入一个油滴,油滴里有结合有引物以及标签的微粒。每个微粒上结合有许多引物,在靠近微粒的一段为逆转录时的引物片段,中间一段为细胞标签 (Cell-barcode),最外侧为单分子计数标签 (Unique Molecule Index, UMI, Molecule-barcode)。一个微粒上都带有同样的细胞标签,用于区分各个油滴里的单细胞,在最初的文献中默认长度为12bp,所以最多可以区别$4^{12}$个细胞,而在通常实验的细胞数量级上,则可以保证几乎没有相同重复的标签。而在单个微粒上,各个细胞标签所连的单分子计数标签又是各不相同的,因为在cDNA扩增时,并非所有的cDNA都可以以相同的倍数扩增,原有的表达量信息经过cDNA扩增后就变得没有意义了,所以需要单分子计数标签用于统计在最初细胞分离时,即每种mRNA连接的不同的单分子计数标签的数目等价于细胞内该种mRNA的数量。

在Drop-seq测序时,一端为标签序列,即细胞标签和单分子计数标签,另一端为mRNA序列。对于测序数据,Drop-seq的开发者提供了一个方便易用的Java包,首先会提取测序一端的细胞标签和单分子计数标签,作为另一端mRNA序列的附加注释。在质控流程中,因为低质量测序的两种标签会影响细胞表达量的统计,而低质量的mRNA序列会影响基因组比对的结果,所以需要分别对测序两端质控。之后再将mRNA比对到基因组上,则这些比对结果根据细胞标签分为各个细胞,每个细胞每个基因的不同单分子计数标签的数目作为这个细胞这个基因的表达量。

\subsection{分析方法}
单细胞RNA测序基于原有RNA-seq技术,可以获取单个细胞的转录组信息,以细胞标签和基因名分别为行或列,以某个细胞某个基因的表达量为值,构成表达矩阵 (Digital Expression Matrix, DEG)。表达矩阵可以用于比较不同细胞的基因表达模式,进行细胞类型的注释,展示不同表达模式的细胞类型在时间和空间上的变化\citep{potter_single-cell_2018}。现在已经有多种快速且方便的分析软件,能辅助处理测序结果和表达矩阵\citep{bacher_design_2016},例如scanpy\citep{wolf_scanpy_2018}、Seurat\citep{stuart_comprehensive_2019}等

\paragraph{数据过滤}
在测序以及对表达矩阵构建的过程中,已经有对测序质量控制的步骤,但仍然有一些细胞中只检测到了很少量的基因表达,这样的细胞一般来说需要被去除。另外,也可以根据线粒体mRNA所占比例来判断这个细胞是否属于低质量细胞。

\paragraph{降维}
未经处理的表达矩阵中的每个细胞可以视为在高维空间中的点,而高维空间中的轴即为不同的基因。为了将这样的复杂数据可视化,就需要将高维空间经过降维变为低维空间。最基本的方法是主成分分析 (Principal Components Analysis, PCA),算法的目的是使用线性变换,将数据点投影到新的坐标轴上,使得每个坐标轴互不相关(协方差为0),并尽量使数据点在坐标轴上投影的方差最大\citep{pearson_liii._1901}。但是在单细胞RNA测序中,维度往往过高,主成分分析并不能很好的将数据点分离开来,所以需要非线性变换处理数据点\citep{becht_dimensionality_2019}。两种常用的非线性降维方法是tSNE\citep{maaten_visualizing_2008}和UMAP\citep{mcinnes_umap:_2018},经过比较,UMAP更能分辨出具有细微差异的细胞类群,并且保持着数据集原有的整体结构,保持细胞类群的连续性,让降维结果更易解读\citep{becht_dimensionality_2019}。但是不可否认的是,在UMAP出现之前,tSNE仍然具有很好的降维效果。

\paragraph{聚类}
经过降维之后,可以在低维空间上将数据点分别划归到各个类群中。常用的聚类方法有Louvian\citep{blondel_fast_2008}、SLM\citep{traag_louvain_2018}等,目的是将空间上邻近的数据点归为一类。而在表达矩阵中,空间上邻近的点即为表达模式相同的点,例如都高表达某个或某一类基因等。

\section{胚胎器官发生转录组分析}
早在2003年,就已经有发育中胰腺的单细胞分析的研究,对60个细胞中的大约100个基因做了分析,发现细胞表达的基因和其可能的分化结果有关\citep{chiang_single-cell_2003},但是在这个研究中仅仅使用了常规微阵列荧光杂交来表示细胞的基因表达情况,由于实验技术的限制,细胞数量和基因数量远少于现在的单细胞RNA测序。在单细胞RNA测序中,可以一次性处理分析大量细胞,并可以覆盖整个基因组的所有已知基因,更容易找到单个细胞类型的表达模式规律\citep{hedlund_single-cell_2018}。

\subsection{构建细胞类型集}
细胞是生命的基本单位,基于细胞的特征对不同细胞的描述、分类显得尤为重要,细胞根据特定基因表达的组合被分为不同的类别。但是在一开始,细胞类型的定义主要根据细胞的位置、形状、细胞组分等,后来利用免疫荧光杂交等技术,但是这些研究都依赖于偶然发现的细胞标志物,导致不同研究队伍使用不同标志物时,难以比较研究结果\citep{regev_human_2017}。所以,在胚胎发育过程中,构建每个器官的细胞类型集可以很大程度上方便对细胞类型的注释,并且也可以统一不同研究者对细胞标志物的选择。

小鼠细胞类型集 (Mouse Cell Atlas, MCA) 是一个可以用于查询在某一组织中,某一基因是否可以作为细胞标志物,若是,则可以获得具体是哪一种细胞类型标志物的网站数据库\citep{han_mapping_2018}。作者对小鼠胚胎各个器官分别取样,再对各个器官的样品进行单细胞RNA测序,一方面对各个组织进行聚类分析,另一方面综合所有表达矩阵,对整个数据集进行聚类分析。作者选择了各个类别有显著差异或具有特异性的表达基因,参照过去证实的各个细胞类型的标志基因,注释各个类别的细胞类型。当整个数据集都被注释完成后,比较不同类型细胞的表达模式,就可以判断一个基因是否在一个类群中显著高表达或者具有特异性,以此建立MCA数据库。

\subsection{解析细胞分化轨迹}
上述提到的细胞类型集的构建仅仅是对于同一时期组织的单细胞RNA测序,以不同时期的样品重复单细胞RNA测序则可以得到不同时期的细胞类型分布,结合时间变化,可以得到一种细胞是如何分化其他细胞类型的,也能得到不同类细胞数量的变化。例如通过表达基因的改变验证神经元沿着放射状胶质细胞迁移过程中的分化过程\citep{zhong_single-cell_2018},或心脏发育过程中先完成心室心房分化再完成左右分化的过程\citep{cui_single-cell_2019},或发现肝脏发育过程中胆管细胞可能是从未成熟肝细胞而不是成肝细胞分化而来\citep{su_single-cell_2017}。因为胚胎器官发生是一个动态的过程,相比于已经成熟的器官,胚胎中未成熟的器官在短时间内细胞类型有着丰富的变化,单细胞RNA测序能准确获得各个时间点的发育情况,综合之后就可以得到动态的器官发生细胞动态分化轨迹。

\section{器官发生研究进展}
自单细胞RNA测序被发明以来,已经有很多器官经过单细胞RNA测序分析之后,人们对其组分的细胞类型有了更深的了解,例如肠\citep{grun_single-cell_2015,haber_single-cell_2017}、大脑\citep{zeisel_cell_2015}、脾脏\citep{jaitin_massively_2014}、胰腺\citep{baron_single-cell_2016,muraro_single-cell_2016}、肺\citep{treutlein_reconstructing_2014}、视网膜\citep{macosko_highly_2015}。但是这些研究使用的都是成熟的器官样品,仅有少数研究针对胚胎发育中的器官发生,下面是有关单细胞RNA测序在胚胎器官发生研究中的一些最新进展。

\subsection{心脏}
心脏是胚胎发育过程中第一个形成的功能性器官\citep{buckingham_building_2005}对人类胚胎心脏发育的单细胞RNA测序将样品细胞分为心肌细胞、内皮细胞、心外膜细胞、成纤维细胞、瓣膜细胞以及一些血细胞,并发现了各类细胞的基因表达特点\citep{cui_single-cell_2019}。作者发现了在心脏发育初期(妊娠第5周)存在心外膜细胞前体细胞高表达心外膜细胞标志物但与成熟的心外膜细胞有不同的特点,并且这些细胞占据了5周时心脏细胞的20\%,所以可能发育为另外种类细胞,对一般观点,即心外膜细胞前体只会发育为心外膜细胞\citep{wessels_epicardium_2004},提出了质疑。
\cite{cui_single-cell_2019}还发现了心脏发育中基因表达模式的时间和空间变化,即心肌细胞在5周时完成心房心室的区分,之后再形成左右心房心室四个腔体。另外,作者也发现了心肌细胞也会表达细胞外基质相关基因,影响细胞增殖、分化、迁移等生命活动\citep{doppler_cardiac_2017},但原有观点认为这些细胞外基质基因只由成纤维细胞表达\citep{lockhart_extracellular_2011},这些细胞外基质基因的非正常表达会导致心力衰竭等疾病\citep{bassat_extracellular_2017},所以该研究结果为心脏发育中细胞外基质基因的研究提供了有效信息。

\subsection{前额叶}
前额叶作为认知功能的中心,是高度专门化的大脑区域\citep{roth_evolution_2012}。即使这些复杂的神经通路往往在胚胎发育末期才形成,但是在发育早期就已经有神经元向合适位置迁移的活动\citep{orahilly_significant_2008}。\cite{zhong_single-cell_2018}使用单细胞RNA测序对人胚胎前额叶分析,将细胞分为35个小类分属于6个大类,即神经前体细胞、兴奋性神经元、中间神经元、星形胶质细胞、少突胶质前体细胞以及小胶质细胞。结合时间分析,作者发现神经前体细胞在妊娠9-10周分化而为胶质细胞谱系和神经元谱系,分别进一步分化为星形胶质细胞、少突胶质细胞和神经元。由于小胶质细胞来源于中胚层\citep{trapnell_dynamics_2014},而中间神经元由神经节隆起迁移而来\citep{ma_subcortical_2013},所以并不包含在分化轨迹内。作者进一步定义了每个小类的基因标志物,在荧光染色的实验中,这些多样细胞类型的新标志物的发现有助于阐明神经回路发育的更精细的细胞基础。另外,作者还发现,受神经营养因子、轴突引导信号等相关基因的调控,中间神经元比兴奋性神经元更晚达到发育高峰期(中间神经元:26周,兴奋性神经元:16周),由于某些神经障碍、社会认知缺陷等疾病与不同类型神经元的比例有关\citep{shi_rin_2005},所以该研究结果有助于进一步认识不同类型神经元的发育情况。

\subsection{消化道}
在妊娠第4周,人类胚胎的原肠可以分为三个部分,前肠、中肠、后肠,进而发育为食道、胃、小肠、大肠四个消化道器官\citep{kraus_patterning_2012}。有一些新生儿疾病与消化道发育异常有关,例如先天性肠闭锁、坏死性小肠结肠炎等\citep{palm_immunoglobulin_2014}。在对消化道四个器官的单细胞RNA测序分析中,\cite{gao_tracing_2018}分别对每个器官进行了细胞类型分类,并定义了每种类别的标记基因。在发育过程上,作者发现食道的细胞类型从妊娠6周开始出现,并直到25周都维持相对不变。胃的各种细胞类型则均匀的分布在妊娠6-25周。不同的是,在6-25周,小肠的发育过程呈现明显的阶段性,发育前期以各类前体细胞为主,后期则以杯状细胞、肠内分泌细胞为主,并且发育前期的细胞增殖相关基因比后期更加活跃。胚胎发育时期的大肠细胞分类显示出和成熟之后的大肠没有差异,二者具有很高的相似性,通过PCA分析并不能区分开来。作者还观察\emph{HOX}基因在不同器官中呈现不同的随时间变化的趋势\citep{beck_homeobox_2002},符合原有的对\emph{HOX}基因表达时空特异性的描述,另外,作者还观察到了存在于人类小肠中的Hedgehog信号通路,与原有的观点保持一致\citep{van_den_brink_indian_2004}。

\subsection{胸腺}
胸腺是主要的淋巴器官,参与建立自身耐受的、功能适应性的免疫系统\citep{gruver_cytokines_2008},发育中的胸腺细胞和胸腺上皮细胞以及血细胞的相互接触诱导了胸腺细胞的选择\citep{hogquist_central_2005}。通过分析小鼠胚胎胸腺单细胞RNA测序的结果,\cite{kernfeld_single-cell_2018}发现自身免疫发病机制可能在胸腺器官发生的早期开始了,并发现了可能与类风湿关节炎的自身免疫性病因有关的基因,并对一个过去认为有关的基因\emph{Plb1}\citep{okada_integration_2014}的有关性提出质疑。这些参与自身免疫的基因在胸腺发育早期表达表明从那个时候起它们已经开始调控胚胎的免疫性。但是在\cite{kernfeld_single-cell_2018}的研究中,预测了可能有关的基因,为后续必要的机理研究提供了有效的信息。另外,在细胞聚类的分析中,\cite{kernfeld_single-cell_2018}观察到了非传统淋巴细胞类群,而这些类群在原有实验基础上是很难观察到的\citep{brennecke_single-cell_2015}。而其他的细胞类型的表达模式展示出了胸腺发育过程中的多层细胞异质性,与已有的知识相吻合\citep{fischer_beyond_2017}。

\subsection{胰腺}
由于在胚胎发育过程中胰腺组织的细胞异质性以及胰腺内分泌细胞的低含量,传统的基因表达分析不能检测到微小的表达差异\citep{saliba_single-cell_2014}。因为在小鼠胚胎发育第13.5天,胰腺上皮层开始了明显的分化并确定各个细胞的谱系\citep{pan_pancreas_2011}。所以通过对13.5天小鼠胚胎胰腺的单细胞RNA测序,\cite{stanescu_single_2017}观测到了数种内分泌细胞的前体细胞,并发现了它们的特异性基因标记以及和成熟细胞的差别。通过伪时间发育轨迹分析,\cite{stanescu_single_2017}发现在发育过程中,内分泌前体细胞的相关基因下调,$\alpha$细胞相关基因上调,表示内分泌前体细胞正在经历向$\alpha$细胞的分化过程作者发现了曾经未被报道过的\emph{Scl38a5, Dbpht},证明它们和胰高血糖素的分泌有关,但由于可能存在的翻译后修饰或者它们不是细胞膜上锚定蛋白,这两个对应的蛋白产物在成熟的$\alpha$细胞中并没有检测到\citep{xin_use_2016}。因为在中枢神经系统中,SLC38A5调节了星形胶质细胞在突触周围细胞膜的甘氨酸和谷氨酸运输\citep{hamdani_system_2012},所以\cite{stanescu_single_2017}认为SLC38A5在未成熟的$\alpha$细胞中行使着同样的功能,并且谷氨酸和甘氨酸都可以调节胰高血糖素和胰岛素的分泌\citep{gao_glucose_1999}。这些新促成熟因子的发现可以用于优化糖尿病的细胞治疗。

\subsection{肝脏}
在胚胎发育早期,肝脏是主要的造血器官,所以大多数肝脏细胞是血细胞\citep{miyajima_stem/progenitor_2014}。除血细胞以外,两种主要的肝脏上皮细胞类型是肝细胞和胆管细胞,它们都由成肝细胞分化而来\citep{gordillo_orchestrating_2015}。
另外,由于成熟的肝脏细胞也保留着分化的潜力,这些肝脏干细胞/前体细胞可以在成熟之后继续分化为肝细胞和胆管细胞\citep{miyajima_stem/progenitor_2014},所以对胚胎发育过程中成肝细胞的研究可以有助于加深对肝脏干细胞/前体细胞的理解。虽然过去的研究发现某些表面膜蛋白参与肝脏干细胞/前体细胞分化命运决定,例如EpCAM、DLK1、CD13\citep{okabe_potential_2009,tanimizu_isolation_2003,kakinuma_analyses_2009}等,但是真正的决定因子还留有争议。因为细胞膜蛋白的表达是一个动态的过程,体外培养又有可能不同于胚胎内发育的真实情况,所以\cite{su_single-cell_2017}对小鼠胚胎肝脏不同时期应用了单细胞RNA测序。数据证明了在胚胎发育第13.5天到16.5天,成肝细胞中参与肝细胞功能的基因表达量逐渐增加,符合原有模型,即受到周围环境Notch、TGF-$\beta$信号通路的调控,成肝细胞会向肝细胞分化\citep{zong_notch_2009}。另一数据表明,由于向肝细胞分化的细胞数目远大于向胆管细胞分化的数目,所以平均来看,会显得成肝细胞只有向肝细胞分化的路径。最后,有数据表明成熟的胆管细胞的基因表达模式更接近于第11.5天的成肝细胞,过去认成肝细胞开始分化的时间是第13.5天\citep{chaudhari_expression_2016},这个问题仍有待解决。
\bibliography{ReviewWriting}
\end{document}
