\documentclass[lang=cn]{elegantpaper}

\title{胚胎器官发生图谱构建中单细胞测序的应用}
\author{赵洵}
\institute{中国科学院大学}
\version{0.01}
\date{\today}

\begin{document}
    \maketitle

    \renewcommand{\abstractname}{摘要}
    \begin{abstract}
        摘要
        \keywords{单细胞测序 器官发育 胚胎发育}
    \end{abstract}

	\section{引言}

    \section{正文}
		\subsection{胚胎器官发生}
        胚胎器官发生指受精卵发育为一个完整个体的过程中,干细胞逐渐分化并形成具有正常功能器官的过程。解析胚胎器官发生的过程有助于区别正常及病理条件下的器官生成机制,理解器官发生过程中的信号通路和不同组织之间的相互作用,以及干细胞分化的机理\citep{belle_tridimensional_2017, lancaster_organogenesis_2014}。对于胚胎器官发育的研究从最初形态学观察研究发展到分子生物学、生物化学染色等手段,一步步从宏观到微观,在细胞、分子层面上展示胚胎器官发生的过程。
        
        受精卵分裂经过原肠胚形成后,变为由三层或两层胚层构成的原肠胚结构,胚层在多种信号通路的控制下继续发育,逐步形成器官\citep{zorn_vertebrate_2009}。
        \subsection{单细胞测序}
        但是在单细胞测序技术出现之前,测序样品局限于细胞群体而非单个细胞,导致了基于测序结果的各类组学分析没有单细胞层面上的分辨率\citep{potter_single-cell_2018}。
            \subsubsection{单细胞转录组测序}
            \subsubsection{单细胞ATAC测序}
	\section{讨论}
        \subsection{讨论第一部分}
            讨论第一部分内容
	\section{结论}
        这是结论
    \bibliography{ReviewWriting}
\end{document}
