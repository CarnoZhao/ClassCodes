\documentclass{ctexart}
\usepackage{geometry, graphicx, float, subfigure, amsmath, bm, threeparttable, multirow, multicol, ifthen}
\usepackage[justification=centering]{caption}
\usepackage[marginal]{footmisc}
\geometry{a4paper,scale=0.8}
\setlength{\parindent}{2em}

\title{\bf{Measure the Probabilities of Conjugation and Transposition Mutagenesis of }$kan^R$ \bf{in E.coli}}
\author{Xun Zhao (Partner: Chase)}
\date{April 19, 2019\\Friday}

\begin{document}
    \begin{titlepage}
        \maketitle
        \setcounter{page}{0}
        \thispagestyle{empty}
    \end{titlepage}

    \renewcommand{\abstractname}{Introduction}
    \begin{abstract}
		The "jumping" gene was dicovered by McClintock who found that the phenotypes of corn's colorful spots sometimes do not follow the Mendelian genetics. She found this kind of genes can move to different place of the genome. Later, it was found that this phenomenon also happens in other species. 
		
		Conjugation is a cell-to-cell contact that can transfer plasmid and the genes on this plasmid from one bacteria cell to another, through a mating bridge called pilus. Transposon is a part of DNA sequence that can move or duplicate itself from one place to another even to different chromosomes. When the transposon is inserted into a gene, it can knock out the old gene and replace its function with new gene's function, which is called transposition mutagenesis.

		The first part of this experiment is to mate two strains of \emph{E.coli}, and let conjugation happens. The second part is using IPTG to induce $kan^R$ transposon on the plasmid to transpose. After doing these, we can calculate the probability of conjugation and transposition mutagenesis by counting the number of certain kind of colonies.

		If conjugation happens, the recipient cell will get the $cm^R$ gene from pVJT128 plasmid, becoming insensitive to chloramphenicol. If transposition happens, the cell will have functional $kan^R$ gene, and become insensitive to kanamycin. If transposition mutagenesis happens at the \emph{lacZ} gene, the colony will be white on X-gel plates, due to it does not have the functional enzyme to digest X-gel.
	\end{abstract}

	\section{Results}
		\subsection{Percent Mutagenesis}
			$$
			\begin{aligned}
			\text{Mutagenesis}\% &= \frac{\#\text{X-gel.Kan.White}}{\#\text{X-gel.Kan.Total}}\\
			&= \frac{3}{((8654 + 3) + 728 \times 10 + 94 \times 100) / 3}\\
			&= \frac{3}{8445} \times 100\%\\
			&\approx 0.035\%
			\end{aligned}
			$$
	\section{Discussion}
			\subsubsection{General Description} 
			The colonies are white and round. The size of colonies varies at the plates with $10 ^ 0$ dilution. This phenomenon indicates that if the plate starts with higher concentration of cells, the colonies are more likely to start from several cells instead of one single cell, which makes the size of colonies different.

			\subsubsection{Percent Conjugation Calculation} 
			\paragraph{Number of Conjugation} There are three kinds of cells that can survive on Cm-Nal plates, the cells after conjugation, the mutant of recipient that is insensitive to Cm, and the mutant of donor that is insensitive to Nal. So the total number of successful conjugation is as follows. Here, I discarded the other three numbers of Cm-Nal plates, since they are too small (2 in $10 ^ {-1}$, 0 in $10^{-2}$, 0 in $10 ^ {-3}$), probably because of the wrong operation in experiment.
			$$
			\begin{aligned}
			\#\text{Conjugation} &= \#\text{Colonies.of.Cm.Nal} - \#\text{Recipient.Mutant} - \#\text{Donor.Mutant}\\
			&= 508 - 0 - 0\\
			&= 508
			\end{aligned}
			$$

			\paragraph{Explanation to Control Plates} Ideally, there if no mutations or contamination happened to recipient and donor cells, they should not survival on Cm plates and Nal plates, respectively. However, in Cm plates of recipient cells, there are 135 colonies. One explanation is that the spontaneous mutation of recipient cells. The second explanation is contamination. Because of the number is much larger than the normal probability of spontaneous mutation. So I decided to omit this part of calculation.

			\paragraph{Total Number of Recipient Cells} This is calculated by the average of three recipient NA plates, and times the dilution $10 ^ {5}$. Because there are only $500 \mu L$ recipient in mating tube but $1000 \mu L$ in recipient tube, the total number should time $\frac{1}{2}$.
			$$
			\begin{aligned}
			\#\text{Total.Recipient} &= \frac{1}{2} \times \frac{946 + 107 \times 10 + 13 \times 100}{3} \times 10 ^{5}\\
			&= 5.525 \times 10 ^ 7
			\end{aligned}
			$$ 
			
			\paragraph{Percent Conjugation} This equals to conjugation number divided by total number.
			$$
			\begin{aligned}
			\text{conjugation}\% &= \frac{508}{5.525 \times 10 ^ 7}\times 100 \%\\
			&\approx 9.2 \times 10 ^ {-4}\%
			\end{aligned}
			$$

			\paragraph{Summary} The percentage indicates that conjugation is a rare occurrence. In my opinion, this might be because the concentration of cells is not so high, causing recipient cell and donor cell cannot contact with each other easily. Moreover, this also might be because conjugation is an energy-consuming behavior, and it is used to recombine genes to deal with extreme environment. So if the environment is suitable for cells, they might be less likely to conjugate with others, instead, they will only multiply themselves.

	\section{Conclusion}
		All of these three biological processes are very rare during the life of bacteria. It fits the idea of inheritance and mutation in genetics, that the genes pass through generation to generation, and keep unchanged. However, there are still low probability that the gene changes, which brings the new traits or new combinations of old traits. 

		The jumping gene indicates that when the environment is not suitable to the bacteria, they know how to recombine their genes with other individuals to survive, and to adapt the new environment. In some way, this is kind of natural selection. The individuals (or populations) with adaptive genes (or traits) can survive and transfer their genes to their offspring, namely, evolution.

		We can use transposon and transposase to bind a new DNA fragment to bacteria's chromosome, which gives us a new method to change the genotype of a bacteria. 
\end{document}
